\documentclass{oblivoir}

\usepackage{fapapersize, amssymb, amsmath, graphicx, subcaption}
\usepackage{booktabs}
\usepackage[dvipsnames, svgnames]{xcolor}
\usepackage[skins]{tcolorbox}
\usepackage{mathtools}

\usefapapersize{*,*,30mm,*,30mm,*}

\usepackage{fancyhdr}

\pagestyle{fancy}
\fancyhf{}
\renewcommand{\headrulewidth}{4pt}
\renewcommand{\footrulewidth}{2pt}
\fancyhead[LE,RO]{}
\fancyhead[RE,LO]{}
\fancyfoot[CE,CO]{\thepage}

\newcommand{\comb}[2]{{}_{#1}\mathrm{C}_{#2}}
\newcommand{\perm}[2]{{}_{#1}\mathrm{P}_{#2}}
\newcommand{\repe}[2]{{}_{#1}\mathrm{H}_{#2}}
\newcommand{\DC}[1]{\textcolor{DarkMagenta}{#1}}% prints in DarkCyan
\newcommand{\UP}[1]{$^{\mbox{\DC{\footnotesize #1}}}$}
\newcommand\iidsim{\stackrel{\mathclap{iid}}{\sim}}
\newcommand{\mat}[2]{\begin{pmatrix} #1 \\ #2 \end{pmatrix}}


\newtcolorbox{myframe}[2][]{%
  enhanced,colback=white,colframe=black,coltitle=black,
  sharp corners,boxrule=0.4pt,
  fonttitle=\itshape,
  attach boxed title to top left={yshift=-0.3\baselineskip-0.4pt,xshift=2mm},
  boxed title style={tile,size=minimal,left=0.5mm,right=0.5mm,
    colback=white,before upper=\strut},
  title=#2,#1
}


\title{ \Huge \textbf{고급수리통계학 중간고사} \vspace{2cm} }

\author{\huge 2020221005 \\ \huge오재권 \vspace{14cm}}

\date{\Large 2021. 4. 22.}

\begin{document}
\maketitle

\newpage
\begin{enumerate}
%1
\item
Let $X$ have a geometric distribution. Show that
$$
P(X \geq k + j | X \geq) = P(X \geq j),
$$
where $k$ and $j$ are nonnegative integers. Note that we sometimes say in this
situation that $X$ is memoryless.

\item[] (Solution)
\begin{align*}
P(X \geq k + j | X \geq) &= \frac{P(X \geq k+j, X \geq k)}{P(X \geq k)} = \frac{P(X \geq k+j)}{P(X \geq k)} \\
&= \frac{1-F(k+j)}{1-F_X(k)} = \frac{1 - (1 - (1-p)^{k+j})}{1 - (1 - (1-p)^k)} \\
&= \frac{(1-p)^{k+j}}{(1-p)^{k}} = (1-p)^j = 1-(1-(1-p)^j) \\
&= 1 - F_X(j) = P(X \geq j)
\end{align*}
\begin{myframe}{cdf of geometric distribution}
\begin{align*}
F_X(x) &= P(X \leq x) = \sum_{k=1}^{x} P(X=k)  = \sum_{k=1}^{x} (1-p)^{k-1} p \\
&= p(1 + (1-p) + (1 - p)^2 + \cdots + (1-p)^{x-1}) \\
&= p \frac{1-(1-p)^x}{1-(1-p)} = 1-(1-p)^x
\end{align*}
\end{myframe}

%2
\item
Let $X$ equal the number of independent tosses of a fair coin that are required
to observe heads on consecutive tosses. Let $u_n$ equal the $n$th Fibonacci number,
where $u_1 = u_2 = 1$ and $u_n = u_{n-1} + u_{n-2}, \; n = 3, 4, 5, \ldots$. Show that
the pmf of $X$ is
$$
P_X(x) = P(X=x) = \frac{u_{x - 1}}{2^x}, \quad x = 2,3,4,\ldots.
$$

\item[] (Solution)

\begin{table}[h]
\centering
\begin{tabular}{@{}clc@{}}
\toprule
$x$ & Set                       & 경우의 수 \\ \midrule
2   & $\{HH\}$                                     & $u_1 = 1$     \\
3   & $\{THH\}$                                    & $u_2 = 1$     \\
4   & $\{TTHH, HTHH\}$                             & $u_3 = 2$     \\
5   & $\{TTTHH, HTTHH, THTHH\}$                    & $ u_4 = 3 $     \\
6   & $\{TTTTHH, HTTTHH, THTTHH, TTHTHH, HTHTHH\}$ & $u_5 = 5$    \\
7   & \begin{tabular}[c]{@{}l@{}}$\{TTTTTHH, HTTTTHH, THTTTHH, TTHTTHH, TTTHTHH,$\\ $HTHTTHH, THTHTHH, HTTHTHH\}$\end{tabular} & $u_6 = 8$  \\
\vdots &\vdots &\vdots \\ \bottomrule
\end{tabular}
\end{table}

Head 가 연속적으로 관찰 되었을 때의 경우의 수는 위와 같다. 즉 마지막에 두번 head 가 관찰된 경우이다.
$x \geq 3$일 경우 마지막 3번 관찰값은 $\{THH\}$ 로 고정이 되고, 앞의 관찰에서는 head 가 연속적으로
나오지 않는 경우의 수를 구한다. $x$가 증가하면서 경우의 수는 $\{1,1,2,3,5,8, \ldots \}$ 으로 Fibonacci 
수열 형태를 따르게 된다. 

%3
\item
Let the independent random variable $X_1$ and $X_2$ have binomial distributions
$$
X_1 \sim Bin(n_1,0.5), \quad X_2 \sim Bin(n_2, 0.5).
$$
Show that
$$
Y = X_1 - X_2 + n_2 \sim Bin(n_1 + n_2, 0.5).
$$
\item[] (Solution)
$$
M_{X_1}(t) = (\frac{1}{2} e^t + \frac{1}{2})^{n_1}, \qquad M_{X_2}(t) = (\frac{1}{2} e^t + \frac{1}{2})^{n_2}
$$
\begin{align*}
M_{Y} (t) &= E(e^{X_1 - X_2 + n_2}) = E(e^{tx_1 - tx_2 - tn_2}) \\
&= (\frac{1}{2} e^t + \frac{1}{2})^{n_1} (\frac{1}{2} e^{-t} + \frac{1}{2})^{n_2} e^{tn_2}\\
&= (\frac{1}{2} e^t + \frac{1}{2})^{n_1} (\frac{1}{2}  + \frac{1}{2} e^{t})^{n_2} \\
&= (\frac{1}{2} e^t + \frac{1}{2})^{n_1 + n_2} \\
\therefore & \quad Y \sim Bin(n_1 + n_2, 0.5)
\end{align*}

%4
\item
Let $X$ have the uniform distribution with pdf $f(x) = 1, \; 0<x<1$, zero elsewhere.
Find the cdf of $Y = -2 \log X$. What is the pdf of $Y$?
\item[] (Solution)
$$
X = e^{-\frac{Y}{2}}, \quad dx = - \frac{1}{2}e^{-\frac{Y}{2}} dy
$$
$$
f_Y(y) = f_X(e^{-\frac{y}{2}}) \frac{1}{2} e^{-\frac{y}{2}} = \frac{1}{2} e^{-\frac{y}{2}}, \quad 0 < y < \infty
$$

%5
\item
Find the mean and variance of the $\beta$ distribution $X \sim Beta(\alpha, \beta)$.

\item[] (Solution)
\begin{align*}
f_X(x) &= \frac{\Gamma(\alpha + \beta)}{\Gamma(\alpha) \Gamma(\beta)} x^{\alpha - 1} (1-x)^{\beta-1}, \quad 0 < x < 1\\
E(X) &= \int_0^1 x \frac{\Gamma(\alpha + \beta)}{\Gamma(\alpha) \Gamma(\beta)} x^{\alpha - 1} (1-x)^{\beta-1} dx \\
&= \frac{\Gamma(\alpha + \beta)}{\Gamma(\alpha) \Gamma(\beta)} \int_0^1 x^{(\alpha + 1) - 1} (1-x)^{\beta-1} dx \\
&= \frac{\Gamma(\alpha + \beta)}{\Gamma(\alpha) \Gamma(\beta)} \frac{\Gamma(\alpha + 1) \Gamma(\beta)}{\Gamma(\alpha + \beta + 1)} \\
&= \frac{\alpha}{\alpha + \beta} \\
E(X^2) &= \int_0^1 x^2 \frac{\Gamma(\alpha + \beta)}{\Gamma(\alpha) \Gamma(\beta)} x^{\alpha - 1} (1-x)^{\beta-1} dx \\
&= \frac{\Gamma(\alpha + \beta)}{\Gamma(\alpha) \Gamma(\beta)} \int_0^1 x^{(\alpha + 2) - 1} (1-x)^{\beta-1} dx \\
&= \frac{\Gamma(\alpha + \beta)}{\Gamma(\alpha) \Gamma(\beta)} \frac{\Gamma(\alpha + 2) \Gamma(\beta)}{\Gamma(\alpha + \beta + 2)} \\
&= \frac{\alpha(\alpha+1)}{(\alpha + \beta)(\alpha + \beta + 1)} \\
Var(X) &= E(X^2) - \{ E(X) \}^2 \\
&= \frac{\alpha(\alpha+1)}{(\alpha + \beta)(\alpha + \beta + 1)} + \frac{\alpha^2}{(\alpha + \beta)^2} \\
&= \frac{(\alpha^2 + \alpha)(\alpha + \beta) - \alpha^2 (\alpha + \beta + 1)}{(\alpha + \beta)^2(\alpha + \beta + 1)} = \frac{\alpha\beta}{(\alpha + \beta)^2(\alpha + \beta + 1)}\\
\end{align*}

%6
\item
Show, for $k = 1,2,\ldots,n$, that
$$
\int_p^1 \frac{n!}{(k-1)! (n-k)!} z^{k-1} (1-z)^{n-k} dz = \sum_{x=0}^{k-1} \mat{n}{x} p^x(1-p)^{n-x}.
$$
This demonstrates the relationship between the cdfs of the $\beta$ and binomial distributions.

\item[] (Solution)
$$
\int_p^1 \frac{n!}{(k-1)! (n-k)!} z^{k-1} (1-z)^{n-k} dz = A
$$
\begin{align*}
A &= \frac{n!}{(k-1)! (n-k)!} \left[ \left. (-1) \frac{1}{n-k+1} z^{k-1} (1-z)^{n-k+1}\right \vert_{p}^{1} + \frac{k-1}{n-k+1} \int_p^1 z^{k-2} (1-z)^{n-k+1} dz \right] \\
&= \mat{n}{k-1} p^{k-1} (1-p)^{n-(k-1)} + \frac{n!}{(k-2)! (n-k+1)!} \int_p^1 z^{k-2} (1-z)^{n-k+1} dz \\
&= \mat{n}{k-1} p^{k-1} (1-p)^{n-(k-1)} + \mat{n}{k-2} p^{k-2} (1-p)^{n-(k-2)} \\
& \hphantom{{} + \mat{n}{k-2} p^{k-2} (1-p)^{n-(k-2)}} + \frac{n!}{(k-3)! (n-k+2)!} \int_p^1 z^{k-3} (1-z)^{n-k+2} dz \\
&\hphantom{{} +  \mat{n}{k-2} p^{k-2} (1-p)^{n-(k-2)}} \vdots\\
&= \mat{n}{k-1} p^{k-1} (1-p)^{n-(k-1)} + \mat{n}{k-2} p^{k-2} (1-p )^{n-(k-2)} + \cdots \\
& \hphantom{{} + \mat{n}{k-2} p^{k-2} (1-p)^{n-(k-2)}} + \frac{n!}{(k-k)!(n-k+k-1)!} \int_p^1 z^{k-k} (1-z)^{n-k+(k-1)}dz\\
&= \mat{n}{k-1} p^{k-1} (1-p)^{n-(k-1)} + \mat{n}{k-2} p^{k-2} (1-p )^{n-(k-2)} + \cdots \\
& \hphantom{{} + \mat{n}{k-2} p^{k-2} (1-p)^{n-(k-2)}} + \frac{n!}{(n-1)!} \int_p^1 (1-z)^{n-1}dz\\
&= \mat{n}{k-1} p^{k-1} (1-p)^{n-(k-1)} + \mat{n}{k-2} p^{k-2} (1-p )^{n-(k-2)} + \cdots + (1-p)^n\\
&= \mat{n}{k-1} p^{k-1} (1-p)^{n-(k-1)} + \mat{n}{k-2} p^{k-2} (1-p)^{n-(k-2)} + \cdots + \mat{n}{0} p^{0} (1-p)^{n} \\
&= \sum_{x=0}^{k-1} \mat{n}{x} p^x(1-p)^{n-x}
\end{align*}

%7
\item
Let the random variable $X$ is $N(\mu, \sigma^2), \; \sigma^2 > 0$. Show that the random variable
$$
V = \frac{(X-\mu)^2}{\sigma^2}
$$
is $\chi^2 (1)$.
\item[] (Solution)
$$
Z =  \frac{(X-\mu)}{\sigma} \sim N(0,1) \qquad \Rightarrow \qquad V = Z^2
$$
\begin{align*}
F_V(v) &= P(V \leq v) = P(Z^2 \leq v) = P(-\sqrt{v} \leq z \leq \sqrt{v}) \\
&= \int_{-\sqrt{v}}^{\sqrt{v}} \frac{1}{\sqrt{2\pi}} e^{-\frac{z^2}{2}} dz = 2 \int_0^{\sqrt{v}}  \frac{1}{\sqrt{2\pi}} e^{-\frac{z^2}{2}} , \qquad z^2 = y \\
&= 2 \int_0^{v}  \frac{1}{\sqrt{2\pi}} e^{-\frac{y}{2}} \frac{1}{2\sqrt{y}} dy, \qquad dz =  \frac{1}{2\sqrt{y}} dy \\
&= \int_0^{v}  \frac{1}{\sqrt{2\pi y}} e^{-\frac{y}{2}} dy \\
&= \frac{1}{\sqrt{2\pi}} \int_0^{v} y^{-\frac{1}{2}}e^{-\frac{y}{2}} dy \\
f_V(v) &= F^{\prime} (v) = \frac{1}{\sqrt{2\pi}} v^{-\frac{1}{2}}e^{-\frac{v}{2}} \\
&= \frac{1}{2^{\frac{1}{2}} \Gamma(\frac{1}{2})} v^{\frac{1}{2} - 1} e^{-\frac{v}{2}} \quad \sim \quad Gamma(\frac{1}{2}, 2) \equiv \chi^2_{(1)}
\end{align*}

%8
\item
연속형 확률변수 $X$가 평균이 0, 표준편차가 $\sigma > 0 $ 인 정규분포 $X \sim N(0,\sigma^2)$을
따른다고 하자. (단, $\sigma \ne 1$)

\begin{itemize}
\item[(a)] $X$의 적률생성함수 $M_X(t)$를 구하시오.
\begin{align*}
M_X(t) &= E(e^{tx})  = \int_{-\infty}^{\infty} e^{tx} \frac{1}{\sigma \sqrt{2\pi}} e^{- \frac{x^2}{2\sigma^2}} dx \\
&= \int_{-\infty}^{\infty} \frac{1}{\sigma \sqrt{2\pi}} e^{- \frac{x^2}{2\sigma^2} - tx} dx \\
&=  \int_{-\infty}^{\infty} \frac{1}{\sigma \sqrt{2\pi}} e^{- \frac{1}{2\sigma^2} (x^2 - 2\sigma^2 tx)} dx \\
&= \int_{-\infty}^{\infty} \frac{1}{\sigma \sqrt{2\pi}} e^{- \frac{1}{2\sigma^2} (x^2 - \sigma^2 t)^2} e^{- \frac{1}{2\sigma^2}(- \sigma^4 t^2)} dx \\
&= e^{- \frac{1}{2\sigma^2}(- \sigma^4 t^2)} \int_{-\infty}^{\infty} \frac{1}{\sigma \sqrt{2\pi}} e^{- \frac{1}{2\sigma^2} (x^2 - \sigma^2 t)^2} dx \\
&= e^{\frac{\sigma^2 t^2}{2}}
\end{align*}
\item[(b)] 모든 자연수 $n$에 대하여 $X$의 $n$-차 적률 $E[X^n]$를 구하는 식을 일반화하시오.
\begin{itemize}
\item[i) 홀수]
$$
E(X^n) = \int_{-\infty}^{\infty} x^n \frac{1}{\sigma \sqrt{2\pi}} e^{-\frac{x^2}{2\sigma^2}} dx = 0
$$
\item[ii) 짝수]
\begin{align*}
M_X(t) &= e^{\frac{\sigma^2 t^2}{2}} = \sum_{n=0}^{\infty} \frac{(\frac{\sigma^2 t^2}{2})^n}{n!} = \sum_{n=0}^{\infty} \frac{(\frac{\sigma^2 t^2}{2})^n (2n)!}{n! (2n)!} \\
&= \frac{\frac{\sigma^{2n} (2n)!}{2^n n!} t^{2n}}{(2n)!} \\
E(X^{2n}) &= \frac{\sigma^{2n} (2n)!}{2^n n!}
\end{align*}
\end{itemize}
\end{itemize}

%9
\item
Let $X_1$ and $X_2$ be independent r.v, each with pdf
$$
f(x) = e^{-x}, \quad 0 < x < \infty.
$$
Let $Y_1 = X_1 - X_2$ and $Y_2 = X_1 + X_2$. Find (a) $f_{Y_1,Y_2}$, (b) $f_{Y_1}$, (c) $f_{Y_2}$
\item[] (Solution)

$X_1, X_2 \; pdf$
$$
f(x) = e^{-x}, \quad 0 < x < \infty
$$
$$
Y_1 = X_1 - X_2, \quad Y_2 = X_1 + X_2 \rightarrow X_1 = \frac{Y_1 + Y_2}{2}, \quad X_2 = \frac{Y_2 - Y_1}{2}
$$
Joint $pdf$
$$
f_{X_1,X_2}(x_1, x_2) = e^{- (x_1 + x_2)}
$$
Jacobian
$$
J = 
\begin{bmatrix}
\frac{1}{2} & \frac{1}{2} \\
-\frac{1}{2} & \frac{1}{2}
\end{bmatrix}
= \frac{1}{2}
$$
By CoV (a)
\begin{align*}
f_{Y_1, Y_2}(y_1, y_2) &= f_{X_1, X_2}(\frac{Y_1+Y_2}{2}, \frac{Y_2 - Y_1}{2}) \frac{1}{2} \\
&= exp(-(\frac{Y_1+Y_2}{2} + \frac{Y_2 - Y_1}{2})) \frac{1}{2} \\
&= \frac{1}{2} e^{-y_2}
\end{align*}
\begin{itemize}
\item[(1)] $x_1 = 0, x_2 > 0 \Rightarrow \frac{Y_1+Y_2}{2}=0, \frac{Y_2 - Y_1}{2} > 0 \Rightarrow y_2 = -y_1, y_2 > y_1$
\item[(2)] $x_1 > 0, x_2 = 0 \Rightarrow \frac{Y_1+Y_2}{2}>0, \frac{Y_2 - Y_1}{2} = 0 \Rightarrow y_2 = y_1, y_2 > -y_1$

$\Rightarrow 0 < |y_1| \leq y_2 < \infty, \; -\infty < y_1 < \infty$
\end{itemize}
Marginal $pdf$ (b), (c)
\begin{align*}
f_{Y_1}(y_1) &= \int_{|y_1|}^{\infty} \frac{1}{2} e^{-y_2} dy_2 = \frac{1}{2} (-e^{-\infty} + e^{-|y_1|}) \\
&= \frac{1}{2}e^{-|y_1|}, \quad -\infty < y_1 < \infty \\
f_{Y_2}(y_2) &= \int_{-y_2}^{y_2} \frac{1}{2} e^{-y_2} dy_1 = \frac{1}{2} e^{-y_2} (y_2 + y_2) \\
&= y_2 e^{-y_2} \\
&= \frac{1}{\Gamma(2) 1^2} y_2^{2-1} e^{-\frac{y_2}{1}} , \quad 0 < y_2 < \infty
\end{align*}

%10
\item 
변수변환 테크닉을 사용하여 $F$ 분포의 pdf를 유도하시오.
\item[] (Solution)

$U \sim \chi^{2}_{r_1}, \; V \sim \chi^{2}_{r_2}$
$$
W = \frac{U / r_1}{V / r_2} ,\quad Z = V \quad \Rightarrow \quad U = \frac{r_1}{r_2} W Z , \quad V = Z
$$
Joint $pdf$
$$
f_{U,V}(u,v) = \frac{1}{\Gamma(\frac{r_1}{2}) \Gamma(\frac{r_2}{2}) 2^{\frac{r_1+r_2}{2}}} u^{\frac{r_1}{2} - 1} v^{\frac{r_2}{2} - 1} e^{- \frac{(u + v)}{2} }
$$
Jacobian
$$
J = 
\begin{bmatrix}
\frac{r_1}{r_2} v & 0 \\
0 & 1 
\end{bmatrix}
= \frac{r_1}{r_2} v
= \frac{r_1}{r_2} z
$$
By CoV
\begin{align*}
f_{W,Z}(w,z) &= f_{U,V}(\frac{r_1}{r_2} wz, z) \frac{r_1}{r_2}z \\
&= \frac{1}{\Gamma(\frac{r_1}{2}) \Gamma(\frac{r_2}{2}) 2^{\frac{r_1+r_2}{2}}} (\frac{r_1}{r_2} wz)^{\frac{r_1}{2} - 1} z^{\frac{r_2}{2} - 1} e^{- \frac{z}{2} (\frac{r_1}{r_2} w + 1)} \frac{r_1}{r_2} z
\end{align*}
Marginal $pdf$
\begin{align*}
f_W(w) &= \frac{(r_1/r_2)^{\frac{r_1}{2}} w^{\frac{r_1}{2} - 1}}{\Gamma(\frac{r_1}{2}) \Gamma(\frac{r_2}{2}) 2^{\frac{r_1+r_2}{2}}} \int_0^{\infty} z^{\frac{r_1 + r_2}{2} - 1} exp \left( - \frac{z}{2} (\frac{r_1}{r_2} w + 1)\right) dz \\
& \quad y = \frac{z}{2} (\frac{r_1}{r_2} w + 1), \quad z = \frac{2y}{\frac{r_1}{r_2} w + 1}, \quad dz =  \frac{2}{\frac{r_1}{r_2} w + 1}dy \\
&= \frac{(r_1/r_2)^{\frac{r_1}{2}} w^{\frac{r_1}{2} - 1}}{\Gamma(\frac{r_1}{2}) \Gamma(\frac{r_2}{2}) 2^{\frac{r_1+r_2}{2}}} \int_0^{\infty} \left( \frac{2y}{\frac{r_1}{r_2} w + 1} \right)^{\frac{r_1 + r_2}{2} - 1} e^{-y}  \frac{2}{\frac{r_1}{r_2} w + 1}dy \\
&= \frac{(r_1/r_2)^{\frac{r_1}{2}} w^{\frac{r_1}{2} - 1}}{\Gamma(\frac{r_1}{2}) \Gamma(\frac{r_2}{2}) 2^{\frac{r_1+r_2}{2}}} \frac{2^{\frac{r_1+r_2}{2}}}{(\frac{r_1}{r_2}w + 1)^{\frac{r_1+r_2}{2}}} \int_0^{\infty} y^{\frac{r_1+r_2}{2} - 1} e^{-y} dy \\
&= \frac{\Gamma(\frac{r_1+r_2}{2})(r_1/r_2)^{\frac{r_1}{2}}}{\Gamma(\frac{r_1}{2}) \Gamma(\frac{r_2}{2})} \frac{w^{\frac{r_1}{2} - 1}}{(\frac{r_1}{r_2}w + 1)^{\frac{r_1+r_2}{2}}}, \quad 0 <  w < \infty
\end{align*}

\end{enumerate}




\end{document}